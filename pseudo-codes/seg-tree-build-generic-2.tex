\begin{algorithm}
    \caption{Build Segment Tree for Range Update and Range Query over Monoid \((S, \ast, e)\)}
    \begin{algorithmic}[1]
        \Procedure{BuildTree}{$A$, $seg$, $lazy$, $n$, $e$}
            \For{$i = 0$ \textbf{to} $2n - 1$}
                \State $seg[i] \gets e$ \Comment{Initialize segment tree with identity element}
                \State $lazy[i] \gets e$ \Comment{Initialize lazy array with identity element}
            \EndFor

            \For{$i = 0$ \textbf{to} $n - 1$}
                \State $seg[n + i] \gets A[i]$ \Comment{Copy array values to leaves of the segment tree}
            \EndFor

            \For{$i = n - 1$ \textbf{downto} 1} \Comment{Build the tree from bottom to top}
                \State $seg[i] \gets \ast(seg[2 \times i], seg[2 \times i + 1])$ \Comment{Combine child nodes}
            \EndFor
        \EndProcedure

        \\

        \Procedure{RangeUpdate}{$seg$, $lazy$, $index$, $l$, $r$, $a$, $b$, $value$, $e$}
            \State \textbf{Call:} \textsc{PushDown}(index, l, r, lazy, e)
            \If{$b < l$ \textbf{or} $a > r$}
                \State \Return \Comment{No overlap, do nothing}
            \ElsIf{$a \leq l$ \textbf{and} $r \leq b$}
                \State $seg[index] \gets \ast(seg[index], (r - l + 1) \times value)$
                \State $lazy[index] \gets \ast(lazy[index], value)$
            \Else
                \State $mid \gets \left\lfloor \frac{l + r}{2} \right\rfloor$
                \State \textsc{RangeUpdate}(seg, lazy, 2 $\times$ index, l, mid, a, b, value, e)
                \State \textsc{RangeUpdate}(seg, lazy, $2 \times index + 1$, mid + 1, r, a, b, value, e)
                \State $seg[index] \gets \ast(seg[2 \times index], seg[2 \times index + 1])$
            \EndIf
        \EndProcedure

        \\

        \Procedure{RangeQuery}{$seg$, $lazy$, $index$, $l$, $r$, $a$, $b$, $e$}
        \State \textbf{Call:} \textsc{PushDown}(index, l, r, lazy, e)
        \If{$b < l$ \textbf{or} $a > r$}
            \State \Return $e$ \Comment{No overlap, return identity element}
        \ElsIf{$a \leq l$ \textbf{and} $r \leq b$}
            \State \Return $seg[index]$ \Comment{Total overlap, return the current node's value}
        \Else
            \State $mid \gets \left\lfloor \frac{l + r}{2} \right\rfloor$
            \State $left \gets$ \Call{RangeQuery}{$seg$, $lazy$, $2 \times index$, $l$, $mid$, $a$, $b$, $e$}
            \State $right \gets$ \Call{RangeQuery}{$seg$, $lazy$, $2 \times index + 1$, $mid + 1$, $r$, $a$, $b$, $e$}
            \State \Return $\ast(left, right)$ \Comment{Combine results from left and right child using \(\ast\)}
        \EndIf
    \EndProcedure

    \Procedure{PushDown}{$index$, $l$, $r$, $lazy$, $e$}
        \If{$lazy[index] \neq e$}
            \State $seg[index] \gets \ast(seg[index], lazy[index])$
            \If{$l \neq r$}
                \State $lazy[2 \times index] \gets \ast(lazy[2 \times index], lazy[index])$
                \State $lazy[2 \times index + 1] \gets \ast(lazy[2 \times index + 1], lazy[index])$
            \EndIf
            \State $lazy[index] \gets e$ \Comment{Clear lazy value after pushing it down}
        \EndIf
    \EndProcedure
    \end{algorithmic}
\end{algorithm}
